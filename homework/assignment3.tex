\documentclass[]{article}
\usepackage[T1]{fontenc}
\usepackage{lmodern}
\usepackage{amssymb,amsmath}
\usepackage{ifxetex,ifluatex}
\usepackage{fixltx2e} % provides \textsubscript
% use microtype if available
\IfFileExists{microtype.sty}{\usepackage{microtype}}{}
\ifnum 0\ifxetex 1\fi\ifluatex 1\fi=0 % if pdftex
  \usepackage[utf8]{inputenc}
\else % if luatex or xelatex
  \usepackage{fontspec}
  \ifxetex
    \usepackage{xltxtra,xunicode}
  \fi
  \defaultfontfeatures{Mapping=tex-text,Scale=MatchLowercase}
  \newcommand{\euro}{€}
\fi
\ifxetex
  \usepackage[setpagesize=false, % page size defined by xetex
              unicode=false, % unicode breaks when used with xetex
              xetex]{hyperref}
\else
  \usepackage[unicode=true]{hyperref}
\fi
\hypersetup{breaklinks=true,
            bookmarks=true,
            pdfauthor={},
            pdftitle={},
            colorlinks=true,
            urlcolor=blue,
            linkcolor=magenta,
            pdfborder={0 0 0}}
\setlength{\parindent}{0pt}
\setlength{\parskip}{6pt plus 2pt minus 1pt}
\setlength{\emergencystretch}{3em}  % prevent overfull lines
\setcounter{secnumdepth}{0}

\author{}
\date{}

\begin{document}

\section{Bios 301: Assignment 3}

\emph{Due Thursday, 21 November, 12:00 PM}

50 points total.

Submit a single knitr (either \texttt{.rnw} or \texttt{.rmd}) file,
along with a valid PDF output file. Inside the file, clearly indicate
which parts of your responses go with which problems (you may use the
original homework document as a template). Raw R code/output or word
processor files are not acceptable.

\subsubsection{Question 1}

\textbf{20 points}

Code a function that does golden section search, and use this function
to find all of the local maxima on the following function:

\[
f(x) = \left\{\begin{array}{lr}0 & \text{if } x=0 \\
|x| \log\left(\frac{|x|}{2}\right)e^{-|x|} & \text{otherwise}
\end{array}\right.\]

on the interval [-10, 10].

To get an idea of what the function looks like, it might be helpful to
plot it.

\subsubsection{Question 2}

\textbf{10 points}

Obtain the code for using Newton's Method to estimate logistic
regression parameters (\texttt{logtistic.r}) and modify it to predict
\texttt{death} from \texttt{weight}, \texttt{hemoglobin} and
\texttt{cd4baseline} in the HAART dataset. Use complete cases only.
Report the estimates for each parameter, including the intercept.

\subsubsection{Question 3}

\textbf{20 points}

Consider the following very simple genetic model (\emph{very} simple --
don't worry if you're not a geneticist!). A population consists of equal
numbers of two sexes: male and female. At each generation men and women
are paired at random, and each pair produces exactly two offspring, one
male and one female. We are interested in the distribution of height
from one generation to the next. Suppose that the height of both
children is just the average of the height of their parents, how will
the distribution of height change across generations?

Represent the heights of the current generation as a dataframe with two
variables, m and f, for the two sexes. The command rnorm(100, 160, 20)
will generate a vector of length 100, according to the normal
distribution with mean 160 and standard deviation 20 (see Section
16.5.1). We use it to randomly generate the population at generation 1:

\begin{verbatim}
pop <- data.frame(m = rnorm(100, 160, 20), f = rnorm(100, 160, 20))
\end{verbatim}

The command \texttt{sample(x, size = length(x))} will return a random
sample of size \texttt{size} taken from the vector \texttt{x}. The
following function takes the data frame \texttt{pop} and randomly
permutes the ordering of the men. Men and women are then paired
according to rows, and heights for the next generation are calculated by
taking the mean of each row. The function returns a data frame with the
same structure, giving the heights of the next generation.

\begin{verbatim}
next_gen <- function(pop) {
    pop$m <- sample(pop$m)
    pop$m <- apply(pop, 1, mean)
    pop$f <- pop$m
    return(pop)
\end{verbatim}

\}

Use the function \texttt{next\_gen} to generate nine generations, then
use the function \texttt{histogram} from the \texttt{lattice} to plot
the distribution of male heights in each generation. The phenomenon you
see is called regression to the mean.

Hint: construct a data frame with variables height and generation, where
each row represents a single man.

\end{document}
