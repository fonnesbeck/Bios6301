\documentclass{beamer}
% preamble
\usetheme{Boadilla}
\definecolor{VandyGold}{RGB}{212,179,125}
\usecolortheme[named=VandyGold]{structure}
\setbeamercolor*{palette primary}{bg=VandyGold,fg=black}
\setbeamercolor*{palette secondary}{fg=VandyGold,bg=black}
\setbeamercolor*{palette tertiary}{bg=VandyGold,fg=black}
\title{From Big to Lite Data with R/SQLite}
\subtitle{Utilizing the SQLite database within R}
\author{Cole Beck}
\institute{Dept of Biostatistics}
\date{\today}
\begin{document}

\begin{frame}
\titlepage
\end{frame}

\begin{frame}
\frametitle{Outline}
\tableofcontents
\end{frame}

\section{Introduction}

\begin{frame}
\frametitle{The Problem}
    A CSV file with one million rows and one thousand columns is a big file.

    $$(1,000,000 * 1,000 * 2) / (2^{30}) = 1.86$$

    Almost 2gb - much bigger than that, and your computer won't be very happy if you create a data.frame by reading it into R.

    \begin{table}
    \begin{tabular}{r|c|c|c|c}
    Row Number & Variable 1 & Variable 2 & ... & Variable 999 \\
    \hline
    1 & 0 & 0 & ... & 1 \\
    2 & 0 & 1 & ... & 0 \\
    3 & 0 & 1 & ... & 1 \\
    4 & 1 & 0 & ... & 0 \\
    5 & 1 & 0 & ... & 1 \\
    \hline
    \hline
    999,999 & 1 & 1 & ... & 1 \\
    \end{tabular}
    \end{table}
\end{frame}

\begin{frame}
\frametitle{Assumptions for Today}
    \begin{itemize}
    \item You have a properly formatted CSV file
    \item You can install SQLite
    \item You have disk space
    \item You only need a subset of the full data
    \item You can identify requested rows by some ID
    \item You want to use R
    \end{itemize}
\end{frame}

\section{Exploration}

\begin{frame}[fragile=singleslide]
\frametitle{Creating Example Files}
\begin{verbatim}
nc <- 1e3 # one thousand
nr <- 5e6 # five million
header <- paste(c('"id"', sprintf('"x%s"', seq(nc))),
                collapse=",")
for(fid in c('A','B','C')) {
    fh <- file(sprintf("bigfile%s.csv", fid), "w")
    writeLines(header, fh)
    ids <- sample(nr)
    for(id in ids) {
        line <- paste(c(sprintf('"%s%s"', id, fid), 
                        sample(0:1, nc, replace=TRUE)),
                    collapse=",")
        writeLines(line, fh)
    }
    close(fh)
}
\end{verbatim}
\end{frame}

\begin{frame}[fragile=singleslide]
\frametitle{~90 minutes later}
\begin{verbatim}
biostat:~/projects/r_workshops/rsqlite$ du -h bigfile*
9.4G    bigfileA.csv
9.4G    bigfileB.csv
9.4G    bigfileC.csv
\end{verbatim}

Just checking the line count takes two minutes

\begin{verbatim}
biostat:~/projects/r_workshops/rsqlite$ wc -l bigfileA.csv
5000001 bigfileA.csv
\end{verbatim}
\end{frame}

\begin{frame}[fragile=singleslide]
\frametitle{Reading Into R}

smallfileA.csv has a more manageable 500,000 rows

\begin{verbatim}
> system.time(x <- read.csv('smallfileA.csv'))
   user  system elapsed 
127.572   3.992 138.254 
> dim(x)
[1] 500000   1001
\end{verbatim}

\begin{verbatim}
> library(data.table)
> system.time(dx <- fread('smallfileA.csv'))
Read 500000 rows and 1001 (of 1001) columns
from 0.936 GB file in 00:00:17
   user  system elapsed 
 15.428   0.868  26.217
\end{verbatim}
\end{frame}

\section{An Alternative}

\begin{frame}
\frametitle{Sequential Read}
The \textbf{readLines} function provides a way to read the full dataset and keep what we want.\\~\\

One option to consider would be creating a row lookup for IDs in each file.\\~\\

How does having the row number help?
    \begin{itemize}
    \item We can utilize \textbf{sed}
    \end{itemize}

This process ties nicely into understanding a database table.
\end{frame}

\begin{frame}[fragile=singleslide]
\frametitle{Row Lookup}
\begin{verbatim}
> system.time({
    nr <- 5e6
    fh <- file('bigfileC.csv', "r")
    readLines(fh, 1)
    idsInFile <- character(nr)
    buffsize <- 10000
    for(i in seq(ceiling(nr/buffsize))) {
        line <- readLines(fh, buffsize)
        a <- (i-1)*buffsize + 1
        b <- min(i*buffsize, nr)
        idsInFile[seq(a,b)] <- sub("[\"](.*)[\"].*", "\\1",
                                    substr(line, 1, 10))
    }
    close(fh)
})
   user  system elapsed 
379.844   3.728 391.45
\end{verbatim}
\end{frame}

\begin{frame}[fragile=singleslide]
\frametitle{Row Lookup}
\begin{verbatim}
> system.time(save(idsInFile, file='idLookup.RData'))
   user  system elapsed 
  4.544   0.012   4.550
> system.time(ix<-match('4321C', idsInFile)); ix
   user  system elapsed 
  1.524   0.348   1.871 
[1] 566498
> cmd <- 'sed "566497p;566498p;566499q;d" bigfileC.csv'
> system.time(system(cmd))
"1321178C",1,0,1,1,0,0,0,1,0,1,1,...
"4192938C",1,1,0,0,1,1,1,1,0,0,1,...
"4321C",0,0,0,0,1,1,1,1,0,1,0,...
   user  system elapsed 
  0.500   0.624  11.381 
\end{verbatim}
\end{frame}

\begin{frame}[fragile=singleslide]
\frametitle{Row Lookup}
sed is sequential - how long would it take to select a line near the end?
\begin{verbatim}
> system.time(system('sed "5000000q;d" bigfileC.csv'))                     
"417927C",0,1,0,1,0,1,0,0,1,1,0,...
   user  system elapsed 
  3.684   5.352  89.814
\end{verbatim}
About 90 seconds - faster than data.table and kinder to your RAM.
\end{frame}

\section{SQLite}

\begin{frame}
\frametitle{Why SQLite?}
    \begin{block}{https://www.sqlite.org/}
    SQLite is a software library that implements a self-contained, serverless, zero-configuration, transactional SQL database engine.
    \end{block}

One of my favourite words is \textbf{zero-configuration}.

    \begin{alertblock}{What could go wrong?}
    2,000 is the maximum number of columns unless you compile it yourself.
    \end{alertblock}
\end{frame}

\begin{frame}[fragile=singleslide]
\frametitle{Loading Into SQLite}
Convert CSV files into a SQLite table.  A 10gb file takes about 20 minutes.
\begin{verbatim}
biostat:~/projects/r_workshops/rsqlite$ sqlite3 mydata.db
sqlite> .mode csv
sqlite> .import bigfileA.csv mytblA
sqlite> .import bigfileB.csv mytblB
sqlite> .import bigfileC.csv mytblC
sqlite> CREATE INDEX ixidA ON mytblA (id);
sqlite> CREATE INDEX ixidB ON mytblB (id);
sqlite> CREATE INDEX ixidC ON mytblC (id);
sqlite> .tables
mytblA  mytblB  mytblC
sqlite> .indices mytblA
ixidA
sqlite> .quit
biostat:~/projects/r_workshops/rsqlite$ du -h mydata.db 
31G     mydata.db
\end{verbatim}
\end{frame}

\begin{frame}
\frametitle{The Key}
Creating the index is extremely important.\\~\\
It deserves its own slide.\\~\\
You can create more than one index per table.\\~\\
It's similar to the row lookup by ID example, except that the search is optimized.
\end{frame}

\begin{frame}[fragile=singleslide]
\frametitle{One Table to Rule Them All}
Optionally collapse the tables into one, if the files are similar.
    \begin{itemize}
    \item It takes longer to create
    \item It won't make the queries run faster
    \item It won't save hard drive space
    \end{itemize}
I wouldn't without a reason
    \begin{itemize}
    \item Unable to differentiate which file a record originates from
    \end{itemize}
\end{frame}

\begin{frame}[fragile=singleslide]
\frametitle{Connecting From R}
Install the \textbf{RSQLite} package
\begin{verbatim}
> library(RSQLite)
> con <- dbConnect(RSQLite::SQLite(), "mydata.db")
> dbListTables(con)
[1] "mytbl"
> dbDisconnect(con)
[1] TRUE
\end{verbatim}
\end{frame}

\begin{frame}[fragile=singleslide]
\frametitle{Running a Query}
What did we gain?
\begin{verbatim}
> system.time({
    a <- dbGetQuery(con, "SELECT id,x1,x2,x1000
            FROM mytbl WHERE id IN 
            ('1234B','4321A','3388C')" )
    })
   user  system elapsed 
  0.004   0.000   0.069 
> a
     id x1 x2 x1000
1 1234B  0  0     1
2 3388C  0  1     0
3 4321A  1  0     1
\end{verbatim}
\end{frame}

\section{A Shiny Ending}
\begin{frame}
\frametitle{A Practical Extension}
This talk has been about extracting a subset a records from a large dataset as quickly as possible.\\~\\
This may only be important in cases where this should be done repeatedly.\\~\\
One practical usage would be an R Shiny app.
\end{frame}

\begin{frame}
\frametitle{Thanks}
Thanks to Nate Mercaldo for sharing the motivating example.
    \begin{block}{This presentation is available on GitHub}
    https://github.com/couthcommander/rsqlite-talk
    \end{block}
\end{frame}

\end{document}
