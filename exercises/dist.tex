\documentclass{beamer}
% preamble
\usetheme{Boadilla}
\definecolor{VandyGold}{RGB}{212,179,125}
\usecolortheme[named=VandyGold]{structure}
\setbeamercolor*{palette primary}{bg=VandyGold,fg=black}
\setbeamercolor*{palette secondary}{fg=VandyGold,bg=black}
\setbeamercolor*{palette tertiary}{bg=VandyGold,fg=black}
\title{nbpMatching R Package}
\subtitle{Optimal non-bipartite matching}
\author{Cole Beck}
\institute{Vanderbilt University}
\date{\today}
\begin{document}

\begin{frame}
\titlepage
\end{frame}

\begin{frame}
\frametitle{Outline}
\tableofcontents
\end{frame}

\section{Distance}
\subsection{Scalar Distance}

\begin{frame}[fragile=singleslide]
\frametitle{scalar.dist function}
Absolute distance between points

\begin{verbatim}
> scalar.dist(1:6)
    [,1] [,2] [,3] [,4] [,5] [,6]
[1,]    0    1    2    3    4    5
[2,]    1    0    1    2    3    4
[3,]    2    1    0    1    2    3
[4,]    3    2    1    0    1    2
[5,]    4    3    2    1    0    1
[6,]    5    4    3    2    1    0
\end{verbatim}
\end{frame}

\subsection{Mahalanobis Distance}

\begin{frame}[fragile=singleslide]
\frametitle{gendistance function}
Distance accounts for covariance matrix

\begin{verbatim}
> gendistance(data.frame(age=c(90,84,91,88,73,82),
    strokes=c(2,1,1,3,1,1)))$dist
        1         2        3        4        5         6
1      Inf 1.2688722 1.402429 1.495769 2.546227 1.4149526
2 1.268872       Inf 1.159632 2.436866 1.822279 0.3313234
3 1.402429 1.1596320      Inf 2.894964 2.981911 1.4909555
4 1.495769 2.4368662 2.894964      Inf 2.744809 2.3946709
5 2.546227 1.8222789 2.981911 2.744809      Inf 1.4909555
6 1.414953 0.3313234 1.490955 2.394671 1.490955       Inf

\end{verbatim}
\end{frame}

\begin{frame}
\begin{block}{Matching}
yay matching
\end{block}

\begin{alertblock}{Warning}
duck!
\end{alertblock}

\begin{definition}
the dictionary defines this as defined
\end{definition}

\begin{example}
this was supposed to be an example
\end{example}

\begin{theorem}{Distance}
$a^2 + b^2 = c^2$
\[
\sqrt{(y_0-y_1)^2 + (x_0-x_1)^2}
\]
\end{theorem}
\end{frame}

\begin{frame}
\frametitle{gendistance function}
gendistance returns a bunch of stuff, in a list

\begin{itemize}
\pause
\item Covariates data set
\pause
\item Distance matrix
\pause
\item things...
\end{itemize}
\end{frame}

\begin{frame}
\label{huge}
This page is \huge{huge}!
\end{frame}

\begin{frame}
\hyperlink{huge}{\beamergotobutton{huge page}}
\end{frame}

\section{Matches}

\setbeamercovered{invisible}
\begin{frame}
\frametitle{Matched Records}

\begin{table}
\begin{tabular}{r|r|r|r|c}
Group1.ID & Group1.Row & Group2.ID & Group2.Row & Distance \\
\hline
\onslide<1,4>{1 & 1 & 4 & 4 & 1.495769} \\
\onslide<2,4>{2 & 2 & 3 & 3 & 1.159632} \\
\onslide<3,4>{5 & 5 & 6 & 6 & 1.490955} \\
\end{tabular}
\caption{Matching Results}
\end{table}

\end{frame}

\end{document}
